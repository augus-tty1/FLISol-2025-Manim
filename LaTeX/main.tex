\documentclass[14pt, aspectratio=169]{beamer}
\usetheme{Copenhagen}
\usecolortheme{spruce}

\usepackage{graphicx}
\usepackage{verbatim}

\title{Manim}
\subtitle{Festival Latino-Americano de Instalação de Software Livre}
\author{Augusto Guerra de Lima}
\date{Abril, 2025}

\begin{document}

\maketitle

%sobre a apresentacao
\begin{frame}{Sobre a apresentação}
\begin{itemize}
    \item<1-> Esta é uma apresentação introdutória
   
    \item<2-> Vamos entender as funcionalidades básicas e fazer exemplos

     \item<3-> Não sou um expert
\end{itemize}
\end{frame}

%o q e manim
\begin{frame}{Sobre Manim}

\begin{itemize}
    \item<1-> Animar conceitos técnicos é tradicionalmente bastante tedioso, pois pode ser difícil torná-los precisos o suficiente para transmiti-los.
    \item<2-> Manim é uma biblioteca que se baseia na simplificade do Python para gerar animações programaticamente.
\end{itemize}

\end{frame}

\begin{frame}{Sobre Manim}
\begin{itemize}
    \item<1-> A biblioteca foi criada por \textbf{Grant Sanderson}. (3b1b)
    \item<2-> Note que, existem duas principais versões. A versão do 3b1b, que se iniciou como um projeto para seu canal no YouTube e a \textbf{Manim community edition} criada em 2020 que é um \textit{fork} da versão do 3b1b; com objetivo de trazer mais estabilidade e testes.
\end{itemize}

\only<2>{Aqui focaremos na versão da comunidade: https://www.manim.community/}

\only<2>{Versão 3b1b: https://github.com/3b1b/manim}
\end{frame}

\begin{frame}{Contribuir com o projeto}
\begin{itemize}
    \item<1->     Página com instruções para contribuição: https://docs.manim.community/en/stable/contributing.html

    \item<2-> 
    Mais recomendado, Discord do Manim community edition: https://discord.com/invite/bYCyhM9Kz2

    \item<3-> Quem pode contribuir ?

    Todos! Manim é grátis e open source; Interessados em matemática, pedagogia, animações no computador, open-source, desenvolvimento de software e muito mais são bem vindos!
\end{itemize}
\end{frame}

%instalacao
\begin{frame}{Instalação}
\begin{itemize}
    \item<1->     Existem muitas formas diferentes de instalar e utilizar Manim. Exemplos incluem: Localmente, ambiente \textit{Conda}, via \textit{Docker} e \textit{Notebooks Jupyter}.
    \item<2-> \textit{Notebooks} são indicados para quem quer testar, por exemplo; se você instalar localmente pode preferir fazer um ambiente virtual com Manim lá dentro ou usar Manim como uma ferramenta global se tiver muitos diretórios e projetos, isso vai depender.
    \item<3-> Tutorial de instalação: https://docs.manim.community/en/stable/installation.html
\end{itemize}
\end{frame}

\begin{frame}[fragile]{Instalação}
    \begin{itemize}
        \item Pode ser útil, para quem quiser testar um pouco no \textbf{Google Colaboratory}, colocar o seguinte bloco de comando:
    \end{itemize}
    \vspace{0.5em}
    \begin{verbatim}
!sudo apt update
!sudo apt install libcairo2-dev \
    texlive texlive-latex-extra texlive-fonts-extra \
    texlive-latex-recommended texlive-science \
    tipa libpango1.0-dev
!pip install manim
!pip install IPython==8.21.0
    \end{verbatim}
\end{frame}

\begin{frame}{Instalação}
    Você vai precisar de instalar:
    \begin{itemize}
        \item \texttt{Python}
        \item \LaTeX (exemplo: \texttt{tex-live}) para escrever coisas como:

        \[e^x = \sum_{n=0}^\infty \frac{x^n}{n!}.\]
        
        \item Manim
    \end{itemize}
    Fortemente recomendado utilizar um ambiente virtual, exemplo: \texttt{uv}.
    https://astral.sh/blog/uv
\end{frame}

\begin{frame}{Cenas e objetos}
    \begin{itemize}
        \item<1->\textbf{Scenes}: As cenas são a tela da animação. O objetivo principal é prover ferramentas para lidar com os objetos, câmeras e animações.
        \item<2->\textbf{Mobjects}: Os objetos matemáticos são efetivamente mostrados na tela. Esses possuem propriedades como posição, dimensão, cor, nome, etc.
        Qualquer objeto que pode ser exibido na tela é um mobject, mesmo que não seja necessariamente de natureza matemática.
        \item <3->Pense em uma peça de teatro e os objetos são os atores.
    \end{itemize}
\only<3>{rodar o \texttt{ex1.py}}
\end{frame}

\begin{frame}[fragile]{Saída e diretórios}
    \begin{verbatim}
        manim -p -qm ex1.py ex1
    \end{verbatim}
    \textbf{Estrutura de diretórios similar:}
\begin{verbatim}
my-project/
├─ex1.py
└─media
  ├─videos
  |  └─scene
  |     └─720p30
  |        ├─ex1.mp4
  |        └─partial_movie_files
    \end{verbatim}
\end{frame}

\begin{frame}[fragile]{Saída e diretórios}
    Você pode produzir múltiplos vídeos usando diversas sessões nas cenas
    \begin{verbatim}
def construct(self):
    # ...
    self.next_section()
    # algumas coisas aqui ...
    self.next_section("nome equivalent") #...
    \end{verbatim}
Mais informações em: {https://docs.manim.community/en/stable/tutorials/
output\_and\_config.html\#sections}
\end{frame}


\end{document}
